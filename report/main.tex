\documentclass[12pt]{article}
\usepackage[italian]{babel}
\usepackage{graphicx}
\usepackage[section]{placeins}
\usepackage{amsmath}
\usepackage{enumitem}
\usepackage{subfigure}
\usepackage{listings}
\usepackage{hyperref}
\usepackage{tcolorbox}

\tcbuselibrary{theorems}

\newtcbtheorem[number within = section]{definition}{\emph{Definition }}{colback = gray!10, colframe = gray!50!black, fonttitle = \bfseries}{th}
\newtcbtheorem[]{theorem}{\emph{Theorem }}{colback = gray!10, colframe = gray!50!black, fonttitle = \bfseries}{th}

\title{Geospatial Data Management Project}
\author{
\begin{tabular}[t]{c@{\extracolsep{8em}}c} 
Giuseppe Maurizio Facchi  & Gregorio Ghidoli \\
mat. 989910 & mat.  \\ 
\end{tabular}
}

\date{A.A. 2021-2022}

\begin{document}
\maketitle
\newpage
\tableofcontents
\newpage
\section{Introduzione}
\begin{definition}{Traiettoria}{}
    Una traiettoria è una traccia generata da un oggetto in movimento in spazi geografici, di solito rappresentata da una serie di punti cronologicamente ordinati $$p_1 \rightarrow p_2 \rightarrow \dots \rightarrow p_n$$ dove ogni punto consiste in una coppia di coordinate geografice associate ad un timestamp $$p=(x,y,t)$$
\end{definition}
Le traiettorie offrono la possibilità di estrarre informazioni da oggetti in movimento, come animali, veicoli o persone.\\
In base alla natura dei dati è possibile classificare le tipologie di dati relative a traiettorie in 4 possibili gruppi (Yu Zheng):
\begin{itemize}
    \item \textbf{Mobilità di persone}
    \item Mobilità di veicoli
    \item Mobilità di animali
    \item Mobilità di fenomeni naturali
\end{itemize}
Il dataset in analisi tratta dati relativi a mobilità di persone, dove 3 visitatori di un museo hanno visto registrare \textbf{attivamente} i loro movimenti per un certo periodo di tempo (da capire) con un RTLS.
\begin{definition}{RTLS}{}
    Si definisce RTLS qualsiasi sistema (hardware + software) in grado di campionare accuratamente la posizione di un'entità in uno spazio definito ad un'alta frequenza.\\
    RTLS non è uno specifico tipo di sistema, ma più un obiettivo che può essere raggiunto utilizzando vari sistemi (Damiani)
\end{definition}
I real-time location systems utilizzati forniscono la posizione di un tag. In base alle caratteristiche del sistema una posizione può indicare:
\begin{itemize}
    \item \textbf{Presenza} in un'area, espressa simbolicamente
    \item \text{Posizione precisa}, espressa in coordinate
    \item \text{Vicinanza} ad un altro tag, espressa come distanza o simbolicamente
\end{itemize}
Nel caso in analisi la posizione è espressa in coordinate con CRS ESPG 3003.\\[12pt]
Si parla in particolare di registrazione \textbf{attiva} (Yu Zheng) quando l'entità tracciata partecipa direttamente all'attività di tracciamento tramite l'utilizzo di RTLS, come nel caso delle persone in analisi.
\newpage
\section{Trajectory Preprocessing}
Prima di analizzare il dataset risulta utile operare operazioni di preprocessing su ogni traiettoria, allo scopo di:
\begin{itemize}
    \item Eliminare/ridurre rumore dovuto all'inaccuratezza del sistema di tracciamento
    \item Identificare \textbf{stop points}
    \item Approssimare la traiettoria stessa, con lo scopo di ridurre la sua dimensione, ma allo stesso tempo non perdere informazione
    \item Segmentare la traiettoria, con lo scopo di estrarre informazioni aggiuntive basate ad esempio sulla forma della traiettoria
\end{itemize}
\subsection{Noise reduction}
\subsection{Stop points detection}
\subsection{Trajectory approximation}
\subsection{Trajectory segmentation}
\section{Trajectory Pattern Mining}
Un importante punto di analisi riguarda lo scoprire gruppi di persone che si muovono insieme per un certo periodo di tempo. Per raggiungere l'obiettivo sono stati sviluppati algoritmi come
\end{document}